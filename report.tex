\documentclass[hyperref,UTF8,titlepage]{ctexbook}
\usepackage[a4paper]{geometry}
\usepackage{amsmath}
\usepackage{bm}
\usepackage{siunitx}
\usepackage{color}
\usepackage{fancyhdr}
\usepackage{float}
%\usepackage{eumeitem}
\usepackage[]{graphicx}
\usepackage{minted}
%\usepackage{pifont} % this is to provide macro \ding in line 12
%
%% \num{10e2}
%% \mmHg
%% \SI{4}{\milli\meter}
%% \gg  % instead of $>\!\!>$
%
%
%%% For footnote style
%\usepackage[perpage]{footmisc}
%\renewcommand\thefootnote{\ding{\numexpr171+\value{footnote}}}

\pagestyle{fancy}
%\geometry{a4paper, bottom=4cm}
%\graphicspath{{}}
\bibliographystyle{plain}

\hypersetup{
        colorlinks=false,
        pdftitle={UGKS读书报告},
        pdfauthor={Z. Y. Liu}}

\newcommand{\me}{\mathrm{e}}
\newcommand{\mi}{\mathrm{i}}
\newcommand{\mRe}{\mathit{Re}}
\newcommand{\mSt}{\mathit{St}}
\newcommand{\mRo}{\mathit{Ro}}
\newcommand{\Karman}{K\'arm\'an}
\newcommand{\diff}{\,\mathrm{d}}

\title{UGKS读书报告}
\author{刘泽宇}
\date{\today}

\begin{document}
\maketitle

\part{背景和原理}
\chapter{问题概述}

\section{CFD 的基本原理}

\section{非平衡流动及其模拟}
在工程实践中, 有许多涉及到稀薄流动的例子, 比如返回式航天飞行器在进入大
气层的过程, 微电子机械系统\footnote{Micro-Electro-Mechanical System,
  MEMS}中的流动,
化学反应伴随的流动, 以及湍流.
在稀薄流动中, 传统的 NS 方程不能很好地描述流体的运动, 甚至存在一些用NS
方程无法解释地现象. NS 方程不能正确描述流动地根本原因在于, 在流动的
Knudsen 数\footnote{Knudsen 是分子自由程和流动特征尺度之比.}不趋向于零时,
流体微团并未达到平衡态, 因此推导出NS方程所使用的基本假设不成立. 从流动
尺度来看, NS方程只在流体动力学尺度上成立, 因为在流体动力学尺度远远大于
分子自由程, 然而在稀薄流动中, 许多流动结构在尺度上是可以和分子自由程相
比拟的, 此时流体运动需要在分子动理学尺度上描述, 因此建立在流体动力学尺
度上的NS方程就难以发挥作用了.

在实际的工程问题中, 流动通常是多尺度的, 因此在流场的不同部分, 流体动力
学平衡态和分子动理学非平衡态往往是共存的, 也就是说, 流动可能会从强
稀薄流逐渐变为连续介质流动. 那么在数目有限到网格中模拟整个流场就变得非
常困难. 在这种条件下, 寻找一种具有内禀多尺度性质的算法就非常重要了.
UGKS就是这样一种通过对流场直接建模构造的算法.

针对稀薄流动的最为成功的算法是 DSMC 方法, 其他的方法包括直接求解
Boltzmann 方程和矩方法.





\chapter{物理学和数学背景知识}

\section{Boltzmann 方程和气体动力学方法}

\subsection{Boltzmann 方程的推导}

\subsection{BGK 模型}

\section{UGKS 的数学基础}

\subsection{BGK 模型的演化解}

\subsection{离散速度空间}

\part{算法构造和分析}
\chapter{基本算法}

\part{应用实例}

\part{杂物}






%% BiBTeX settings
\bibliography{report}
\end{document}

%% The simple manual for minted package
\begin{minted}
  [
    frame=lines,
    framesep=2mm,
    baselinestretch=1.2,
    bgcolor=LightGray,
    fontsize=\footnotesize,
    linenos
  ]
  {python}
  % some Python code
\end{minted}

\inputminted[firstline=2, lastline=12]{octave}{BitXorMatrix.m}

\mint{html}|<h2>Something <b>here</b></h2>|

%% End %% Actually there are many fabulous features...
